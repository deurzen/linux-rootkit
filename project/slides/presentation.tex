\documentclass{i20lecture}
\usepackage{listings}

\subtitle{LiveDM --- Proof of Concept}

\begin{document}

\frame{\titlepage}

%%%%%%%%%%%%%%%%%%%%%%%%%%%%%%%%%%%%%%%%%%%%%%%%%%%%%%%%%%%%%%%%%%%%%%%%%%%%%%%%%%%%%%%%%%%%%%%%%%%%%%%%%%%%%%
\section{Agenda}
%%%%%%%%%%%%%%%%%%%%%%%%%%%%%%%%%%%%%%%%%%%%%%%%%%%%%%%%%%%%%%%%%%%%%%%%%%%%%%%%%%%%%%%%%%%%%%%%%%%%%%%%%%%%%%
\begin{frame}{\insertsection}
  \begin{enumerate}
   \item Background
    \begin{itemize}
        \item Dynamic Kernel Memory
        \item LiveDM
    \end{itemize}
    \item Approach
    \item Results
    \item Discussion / Questions
  \end{enumerate}
\end{frame}

%%%%%%%%%%%%%%%%%%%%%%%%%%%%%%%%%%%%%%%%%%%%%%%%%%%%%%%%%%%%%%%%%%%%%%%%%%%%%%%%%%%%%%%%%%%%%%%%%%%%%%%%%%%%%%
\section{Background}
%%%%%%%%%%%%%%%%%%%%%%%%%%%%%%%%%%%%%%%%%%%%%%%%%%%%%%%%%%%%%%%%%%%%%%%%%%%%%%%%%%%%%%%%%%%%%%%%%%%%%%%%%%%%%%
\begin{frame}{\insertsection}
  \framesubtitle{Dynamic Kernel Memory}

  \begin{itemize}
    \item Dynamic kernel memory is...
    \begin{itemize}
\pause
    \item ...hard to make sense of -- usually, no type information is available
\pause
	\item ...constantly changing -- it's dynamic, after all
\pause
    \item ...difficult to analyze!
    \end{itemize}
\pause
    \item How can we make analysis easier?
  \end{itemize}
\end{frame}

\begin{frame}{\insertsection}
  \framesubtitle{LiveDM --- Overview}

  \begin{itemize}
	\item LiveDM seeks to overcome these issues through Virtual Machine Introspection (VMI)
	\pause
	\begin{itemize}
		\item Monitor the runtime state of a system-level VM
		\pause
		\item Without altering the guest OS
	\end{itemize}
\pause
    \item Memory allocation events can be intercepted
\pause
    \item Going from there, LiveDM is able to create a memory map
	\begin{itemize}
		\pause
		\item This map includes type information!
	\end{itemize}
  \end{itemize}
\end{frame}

\begin{frame}{\insertsection}
  \framesubtitle{LiveDM --- Overview}

  \begin{itemize}
    \item Three distinct stages to create the mapping:
    \begin{enumerate}
\pause
     \item Gathering of necessary values
\pause
     \item Determining the scope of memory monitoring
\pause
     \item Performing type interpretation
    \end{enumerate}
  \end{itemize}
\end{frame}

\begin{frame}{\insertsection}
  \framesubtitle{LiveDM --- Stage 1}

  \begin{itemize}
    \item Stage 1 is comprised of...
    \begin{itemize}
     \item ...intercepting a set of memory allocation/deallocation functions
\pause
     \item ...retrieving the requested allocation size, as well as the return value
\pause
	 \item ...identifying the caller (call site) through the stack's return address
    \end{itemize}
  \end{itemize}
\end{frame}

\begin{frame}{\insertsection}
  \framesubtitle{LiveDM --- Stage 2}

  \begin{itemize}
    \item In stage 2, the scope of memory monitoring is chosen
    \begin{itemize}
\pause
     \item Offer snapshots of the memory map (containing type and size for allocated memory)
     \begin{itemize}
		 \item We offer this in our PoC (\lstinline|rk-print-mem| and \lstinline|rk-data <address>|)
     \end{itemize}
\pause
	 \item Trace every memory (write) access on known (vulnerable) memory blocks
     \begin{itemize}
      \item Warn when critical values are written to traced blocks
      \item We are able to showcase this in a small demo
     \end{itemize}
    \end{itemize}
  \end{itemize}
\end{frame}

\begin{frame}{\insertsection}
  \framesubtitle{LiveDM --- Stage 3}

  \begin{itemize}
    \item In stage 3, the caller's address is translated into a type
\pause
    \begin{itemize}
     \item Relies on instrumenting GCC to retrieve abstract syntax tree (AST)
    \end{itemize}
  \end{itemize}
\end{frame}

\begin{frame}{\insertsection}
  \framesubtitle{LiveDM --- Motivation}

  \begin{itemize}
    \item Why do we need this information? Possible answers include...
    \begin{enumerate}
\pause
     \item To make dynamic memory less transparent
\pause
     \item To utilize this information for debugging
\pause
     \item \textcolor{yellow}{To utilize this information for rootkit detection}
    \end{enumerate}
  \end{itemize}
\end{frame}

%%%%%%%%%%%%%%%%%%%%%%%%%%%%%%%%%%%%%%%%%%%%%%%%%%%%%%%%%%%%%%%%%%%%%%%%%%%%%%%%%%%%%%%%%%%%%%%%%%%%%%%%%%%%%%
\section{Approach}
%%%%%%%%%%%%%%%%%%%%%%%%%%%%%%%%%%%%%%%%%%%%%%%%%%%%%%%%%%%%%%%%%%%%%%%%%%%%%%%%%%%%%%%%%%%%%%%%%%%%%%%%%%%%%%
\begin{frame}{\insertsection}
  \begin{enumerate}
   \item Background
    \item Approach
    \begin{itemize}
        \item Tools
        \item Implementing stage 1 --- Gathering of necessary values
        \item Implementing stage 3 --- Performing type interpretation
        \item Implementing stage 2 --- Determining the scope of memory monitoring
    \end{itemize}
    \item Results
    \item Discussion / Questions
  \end{enumerate}
\end{frame}

\begin{frame}{\insertsection}
  \framesubtitle{Tools}

  \begin{itemize}
   \item Since introspection techniques are required, we need a VMM
\pause
    \begin{itemize}
    \item Xen
    \item KVM
	\item \textcolor{yellow}{QEMU} (in vivo introspection using GDB)
    \item ...
    \end{itemize}
  \end{itemize}
\end{frame}

\begin{frame}{\insertsection}
  \framesubtitle{Implementing stage 1 --- Gathering of necessary values}

    \begin{itemize}
	\item Intercepting allocations is easy: non-blocking \textbf{breakpoints}
     \begin{itemize}
\pause
      \item Break on function entry and exit
        \begin{itemize}
\pause
			\item At entry we extract allocation size
			\item At exit we extract return value (base address of allocation)
        \end{itemize}
\pause
      \item Has a significant performance overhead, but system is still usable
\pause
      \item Possible improvement: hardware breakpoints
\pause
        \begin{itemize}
			\item Limited to a small amount
        \end{itemize}
     \end{itemize}
    \end{itemize}
\end{frame}

\begin{frame}[fragile]{\insertsection}
  \framesubtitle{Implementing stage 1 --- Gathering of necessary values}

    \begin{itemize}
     \item To retrieve the size and return value of each allocation, we can rely on the System V calling convention
\pause
     \begin{itemize}
      \item As the size is not always the first argument, we build a dictionary:
     \end{itemize}
    \end{itemize}
    \begin{lstlisting}
    break_arg = {
        "kmem_cache_alloc_trace": "rdx",
        "kmalloc_order": "rdi"
        [...]
    }
    \end{lstlisting}
\pause
	\begin{itemize}
	\item[]
	\begin{itemize}
		\item Return values are gathered by additionally breaking on return instructions
		\pause
		\begin{itemize}
			\item Look for \lstinline|ret{,q}| instruction's offset from function entry in the disassembly
			\pause
			\item Break on \lstinline|<function entry> + <ret offset>|
			\pause
			\item Retrieve return value from \lstinline|$rax|
		\end{itemize}
	\end{itemize}
	\end{itemize}
\end{frame}

\begin{frame}[fragile]{\insertsection}
  \framesubtitle{Implementing stage 3 --- Performing type interpretation}
    \begin{itemize}
     \item Translation of call sites to types
	 \pause
     \item Possible approaches:
     \begin{itemize}
\pause
      \item Instrumenting \lstinline|gcc| to extract AST (LiveDM)
\pause
      \item Use \lstinline|clang| to generate an AST without instrumentation
\pause
	  \item \textcolor{yellow}{Utilize GDB's \lstinline|whatis| command to statically pre-compute type dictionary}
     \end{itemize}
    \end{itemize}
\end{frame}

\begin{frame}[fragile]{\insertsection}
  \framesubtitle{Implementing stage 3 --- Performing type interpretation}
    \begin{itemize}
     \item Process for generating the type dictionary: \footnote{Fully automated, since specific to kernel sources version, build options, and compiler optimizations}
\pause
     \begin{enumerate}
      \item Find all occurences of function calls we are interested in using \lstinline|cscope|
\pause
      \item Iterate the generated occurences using Python; execute \lstinline|whatis| on every assigned-to symbol
      \begin{itemize}
\pause
       \item Assumption: debug symbols for current kernel sources are available
\pause
       \item Compound type access chains (e.g., \lstinline|desc->inbuf|) have to be recursively resolved
	\pause
	   \item We only require the type of the last dereferenced field, as that is what's being assigned to
      \end{itemize}
\pause
      \item Store the results in a dictionary
\pause
      \item Use this precompiled type dictionary in our runtime script
\pause
     \end{enumerate}
    \end{itemize}
    \begin{lstlisting}
     "./arch/x86/kernel/e820.c:675": "type = struct e820_table *",
     "./arch/x86/kernel/e820.c:681": "type = struct e820_table *"
    \end{lstlisting}
\end{frame}

\begin{frame}[fragile]{\insertsection}
  \framesubtitle{Implementing stage 3 --- Performing type interpretation}
    \begin{itemize}
     \item Once a breakpoint is encountered, we can walk the stack with gdb...
    \end{itemize}
\pause
    \begin{lstlisting}
    #0  __kmalloc (size=168, flags=6291456) at ./mm/slub.c:3784
    #1  0xffffffffa9384095 in kmalloc (flags=<optimized out>, size=<optimized out>) at ./include/linux/slab.h:520
    #2  bio_alloc_bioset (gfp_mask=6291456, nr_iovecs=<optimized out>, bs=0x0) at ./block/bio.c:452
    \end{lstlisting}
\pause
    \begin{itemize}
     \item ...and match the \lstinline|file:line| descriptor to a type without expensive computations
    \end{itemize}
\end{frame}

\begin{frame}[fragile]{\insertsection}
  \framesubtitle{Implementing stage 2 --- Determining the scope of memory monitoring}

    \begin{enumerate}
     \item Snapshot-based approach
\pause
     \begin{itemize}
      \item Since we already store everything gathered, this is readily available
\pause
      \item Live allocations can be listed with \lstinline|rk-print-mem| and interpreted with \lstinline|rk-data <address>|:
     \end{itemize}
    \end{enumerate}
    \begin{lstlisting}
    > rk-print-mem
	  type: struct task_struct *, size: 3776 B, address: 0xffff8e72b87ce740, call site: ./kernel/fork.c:807
	  type: struct fdtable *, size: 56 B, address: 0xffff8e72b84104c0, call site: ./fs/file.c:111
    \end{lstlisting}
    \begin{lstlisting}
	> rk-data 0xffff8e72b84104c0
	  resolving 0xffff8e72b84104c0 to type = struct fdtable *

	  $17 = {
	    max_fds = 256,
	    fd = 0xffff8e72b8ea4800,
	    close_on_exec = 0xffff8e72b8411800,
	    open_fds = 0xffff8e72b84117e0,
	    full_fds_bits = 0xffff8e72b8411820,
	    rcu = {
	  	next = 0x0,
	  	func = 0x0
	    }
	  }
    \end{lstlisting}
\end{frame}

\begin{frame}[fragile]{\insertsection}
  \framesubtitle{Implementing stage 2 --- Determining the scope of memory monitoring}

    \begin{enumerate}
     \setcounter{enumi}{1}
     \item Memory-access tracing
\pause
    \begin{itemize}
     \item Would require some advanced techniques (e.g., page unmapping) for full coverage
\pause
    \item Not feasible within the given time frame
\pause
    \item Instead, we will demonstrate a small example later based on \textit{hardware} watchpoints
    \end{itemize}
    \end{enumerate}
\end{frame}

%%%%%%%%%%%%%%%%%%%%%%%%%%%%%%%%%%%%%%%%%%%%%%%%%%%%%%%%%%%%%%%%%%%%%%%%%%%%%%%%%%%%%%%%%%%%%%%%%%%%%%%%%%%%%%
\section{Results}
%%%%%%%%%%%%%%%%%%%%%%%%%%%%%%%%%%%%%%%%%%%%%%%%%%%%%%%%%%%%%%%%%%%%%%%%%%%%%%%%%%%%%%%%%%%%%%%%%%%%%%%%%%%%%%

\begin{frame}[fragile]{\insertsection}
  \framesubtitle{Demo 1 - Allocation \& Deallocation}
    \begin{itemize}
     \item We will demonstrate the output in a running system now:
     \end{itemize}
     \begin{lstlisting}
      Allocating ('type = struct elf64_phdr *', 616, './fs/binfmt_elf.c:441') at 0xffff8d96b8857000
      Allocating ('type = char *', 28, './fs/binfmt_elf.c:762') at 0xffff8d96ba5d98e0
      Allocating ('type = struct elf64_phdr *', 504, './fs/binfmt_elf.c:441') at 0xffff8d96bb4b1e00
      Allocating ('type = void *', 168, './block/bio.c:452') at 0xffff8d96ba14bcc0

     \end{lstlisting}

\end{frame}

\begin{frame}[fragile]{\insertsection}
  \framesubtitle{Demo 2 - Rootkit Detection}
    \begin{itemize}
     \item We will demonstrate the rootkit detection in a running system now:
     \end{itemize}
     \begin{lstlisting}
        //inside the vm, rootkit is loaded
        > make_me_root
     \end{lstlisting}
     \begin{lstlisting}
        ((((struct task_struct *)0xffff8d96bb6849c0)->real_cred)->uid) changed from val = 1000 to val = 0
        WARNING: critical value 0 set to ((((struct task_struct *)0xffff8d96bb6849c0)->real_cred)->uid)

     \end{lstlisting}
     

\end{frame}

%%%%%%%%%%%%%%%%%%%%%%%%%%%%%%%%%%%%%%%%%%%%%%%%%%%%%%%%%%%%%%%%%%%%%%%%%%%%%%%%%%%%%%%%%%%%%%%%%%%%%%%%%%%%%%
\section{Discussion / Questions}
%%%%%%%%%%%%%%%%%%%%%%%%%%%%%%%%%%%%%%%%%%%%%%%%%%%%%%%%%%%%%%%%%%%%%%%%%%%%%%%%%%%%%%%%%%%%%%%%%%%%%%%%%%%%%%
\begin{frame}{\insertsection}
  \begin{center}
    \LARGE \dots
  \end{center}
\end{frame}



\end{document}
